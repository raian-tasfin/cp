\documentclass[../../main.tex]{subfiles}


\begin{document}
\chapter{Preface}

These are notes on our study of `Algorithms and Data Structures'. How
will we keep rabbit holes of other fields away is still
undecided. Without delving into the debate of the boundaries of
different fields, what we can try is to keep things neat and keep the
theme consistent. One rabbit hole we can think of is that of `formal
methods'. We have to be cautious from the earliest sections to avoid
that one. For now, we have the following motivations for studying
these

\begin{enumerate}[itemsep=-0.5em, topsep=0pt]
\item Academic theory course.
\item Academic lab course.
\item Programming contest.
\item Personal curiosity.
\end{enumerate}

\noindent
The list is sorted in descending order of urgency. Let us shed a bit
more light on them



\subsection{Academic Theory Course(s)}
We have previously gone through CSE 211 which was titled Data
Structures. At the moment of writing this we have CSE 411 titled
`Design and Analysis of Algorithms'. We have to naturally follow our
instructors' directions for the duration of the course. We are far
past 211, and are left with 411 right now. For 411 the instructor
prefers \cite{clrs4} but does not mind older editions (specially
\cite{clrs3}).


\subsection{Academic Lab Course}
We have two lab course instructors. While both are liberal about the
resources, one recommends \cite{Aspnes}. I am a huge fan too. But this
is mostly focused on implementation (we are of course talking about the
lab course). The C review is also appealing.


\subsection{Programming contests}
We of course do have a liking towards competitive programming, and for
that require good knowledge on the topic. More importantly, we need
lists of problems on each topic. For problems on relevant topics, we
may refer to \cite{cpalgo}, \cite{shohagTopics}.


\subsection{Personal Curiosity}
We are of course highly interested in these topics. However, our
interest is equally balanced between correctness, complexity and
implementation. If we dive too deep into the correctness rabbit hole,
we may discover ourselves in the realm of `Formal Methods'. In this
document we will restrain ourselves against that.


\section{Pedagogic Approach}
The thing is, most of the motivations behind these studies do not
require much rigor except our personal preference. Hence we will first
look at the big picture.


\end{document}
